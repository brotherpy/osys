\chapter{Unidad V.}
\section{
	Programas Maliciosos
}
Los programas maliciosos, también conocidos como malware, son software diseñados con el propósito de dañar, acceder de manera no autorizada o comprometer el funcionamiento normal de un sistema informático. Estos programas maliciosos pueden afectar a computadoras, dispositivos móviles, redes y otros sistemas. 
\subsection{Características}
Existen varios tipos de programas maliciosos, sin embargo comparten algunas características comunes que lo identifican:

\begin{tcolorbox}[title= Características de los programas maliciosos]
	\begin{itemize}
		
		
		\item \textbf{Auto-replicación:}
		Algunos tipos de malware, tienen la capacidad de replicarse y propagarse de un sistema a otro. Esto les permite extenderse rápidamente a través de redes y dispositivos.
		\item \textbf{Camuflaje:}
		Los programas maliciosos a menudo intentan ocultar su presencia para evitar ser detectados. 
		\item \textbf{Daño o alteración:}
		Muchos tipos de malware tienen la capacidad de dañar archivos, sistemas operativos o incluso hardware. Pueden borrar, corromper o modificar datos críticos, lo que puede afectar el rendimiento y la integridad del sistema.
		\item \textbf{Robo de información:}
		Muchos programas maliciosos, están diseñados para recopilar información confidencial del usuario, como contraseñas, datos bancarios y detalles personales.
		\item \textbf{Polimorfismo:}
		Algunos programas maliciosos tienen la capacidad de cambiar su apariencia y comportamiento con el tiempo, lo que dificulta su detección por parte de programas antivirus que dependen de firmas estáticas.
		
	\end{itemize}
\end{tcolorbox}
\subsection{Tipos de programas maliciosos}
\begin{tcolorbox}
	\begin{itemize}
		
		
		\item \textbf{Virus:}
		 Un virus es un programa que se inserta en el código de otro software y se propaga cuando ese software se ejecuta.
		\item \textbf{Gusanos (Worms):}
		A diferencia de los virus, los gusanos no necesitan infectar otros programas para propagarse; se replican y se envían a través de redes.
		\item \textbf{Troyanos (Trojans):}
		Estos programas maliciosos se disfrazan como software legítimo para engañar a los usuarios y obtener acceso no autorizado a sus sistemas.
		\item \textbf{Spyware:}
		Su objetivo principal es recopilar información sobre las actividades de un usuario sin su conocimiento, como contraseñas, historial de navegación y datos personales.
		\item \textbf{Adware:}
		Diseñado para mostrar anuncios no deseados en el sistema infectado, a menudo generando ingresos para los creadores del malware.
		\item \textbf{Rasomware:}
		Bloquea el acceso a archivos o el sistema completo y exige un rescate para restaurar el acceso.
		\item \textbf{Rootkits:}
		 Estos programas se instalan de manera sigilosa y ocultan su presencia al sistema operativo, permitiendo a los atacantes mantener un acceso no autorizado.
	\end{itemize}
\end{tcolorbox}
\subsection{Medios de propagación de los programas maliciosos}
Los programas maliciosos se propagan a través de diversos medios y técnicas. 
Algunos de los medios de propagación comunes incluyen:
\begin{tcolorbox}[title= Medios de propagación de los programas maliciosos]
	\begin{itemize}
		
		
		\item \textbf{Correos electrónicos de phishing:}
		Los correos electrónicos fraudulentos intentan engañar a los usuarios para que abran archivos adjuntos o hagan clic en enlaces maliciosos. 
		\item \textbf{Sitios web maliciosos:}
		 Los sitios web comprometidos o específicamente creados para distribuir malware pueden aprovechar vulnerabilidades en los navegadores web o utilizar técnicas de descarga automática para infectar los sistemas de los visitantes. 
		\item \textbf{Dispositivos de almacenamiento extraíbles:}
		 Los programas maliciosos pueden propagarse a través de dispositivos de almacenamiento USB u otros medios extraíbles cuando se conectan a sistemas comprometidos. 
		\item \textbf{Vulnerabilidades de software:}
		Los programas maliciosos pueden aprovechar vulnerabilidades en el sistema operativo o en el software instalado para infectar el objetivo. 
		\item \textbf{Conexiones de red no seguras:}
		La propagación puede ocurrir a través de redes no seguras, especialmente en entornos Wi-Fi públicos, donde los atacantes pueden interceptar y modificar el tráfico de red.
		
	\end{itemize}
\end{tcolorbox}
\subsection{Macrovirus}
Un macrovirus es un tipo de virus informático que utiliza macros, que son secuencias de comandos automatizados, para llevar a cabo acciones maliciosas en documentos y aplicaciones que admiten la ejecución de macros(Word, Excel). Los macros son conjuntos de instrucciones que pueden automatizar tareas específicas dentro de documentos o programas.
\subsubsection{ ¿Cómo funciona un macrovirus?
}
\begin{tcolorbox}[title= Funcionamiento de un macrovirus]
	\begin{enumerate}
		
		
		\item \textbf{Infección inicial:}
		El macrovirus generalmente se inserta en un documento, como un archivo de Microsoft Word o Excel, que admite macros.  
		\item \textbf{Ejecución de macros:}
		Cuando el usuario abre el documento infectado, los macros maliciosos intentan ejecutarse automáticamente. En algunos casos, el usuario puede verse inducido a habilitar la ejecución de macros, creyendo que es necesario para ver el contenido completo del documento.
		\item \textbf{Acciones maliciosas:}
		Una vez que los macros se activan, realizan acciones maliciosas diseñadas por los creadores del virus. 
		\item \textbf{Persistencia:} 	
		Algunos macrovirus están diseñados para ser persistentes, lo que significa que intentarán mantenerse en el sistema incluso después de que el documento original se cierre o se elimine.
	
		
	\end{enumerate}
\end{tcolorbox}
